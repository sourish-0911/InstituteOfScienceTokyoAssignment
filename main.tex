\typeout{IJCAI-19 Multiple authors example}
\documentclass{article}
\pdfpagewidth=8.5in
\pdfpageheight=11in
% The file ijcai19.sty is NOT the same than previous years'
\usepackage{ijcai19}
% Use the postscript times font!
\usepackage{times}
\usepackage{soul}
\usepackage{url}
\usepackage[hidelinks]{hyperref}
\usepackage[utf8]{inputenc}
\usepackage[small]{caption}
\usepackage{graphicx}
\usepackage{amsmath}
\usepackage{booktabs}
\urlstyle{same}
% the following package is optional:
%\usepackage{latexsym} 

% Following comment is from ijcai97-submit.tex:
% The preparation of these files was supported by Schlumberger Palo Alto
% Research, AT\&T Bell Laboratories, and Morgan Kaufmann Publishers.
% Shirley Jowell, of Morgan Kaufmann Publishers, and Peter F.
% Patel-Schneider, of AT\&T Bell Laboratories collaborated on their
% preparation.

\title{My dream is to build - Something Something . . .}

\author{
Sourish Chatterjee$^1$\footnote{chatterjee.sourish@takenaka.co.jp}}
\begin{document}

\maketitle

\begin{abstract}
In about 150 - 250 words write the abstract of your dream project. If it is feasible for you to quantify the impact of your dream project by doing some internet search it will be welcomed. 
\par
\textbf{General advice} --- This is valuable self introduction that will help the reader to develop interest in the further paper. Its is advisable to keep the highlights only, details can be added in further sections.
\end{abstract}

\section{What is your Dream Project?}
Introduce your dream project here. Use at max 2 paragraphs to tell details about your dream. Include the following details.
\par
Feel free to add the general principals of your dream project.
Include details like what is currently available solution closest to your dream project. 

\section{Why does your Dream Project matters?}
In one paragraph or a series of bullet points tell us why does your dream matters if there are some available solution closer to your dream project. Highlight the main differentiating factors between your project versus the available solutions. Also take the chance to highlight the gaps you would like to fulfill. Define your goals using the highlighted gaps. Please feel free to cite any references \cite{texbook}

\section{What action you want to take to fulfill your Dream Project?}
Now it is time to get real. Do tell us in few bullet points what are the steps you will take to achieve the goals in the previous section. The steps can be simple but a logical sequence should arise from the steps you would like to take.

\section{Conclusion}
This part of the whole write up is your opportunity to give a closing remark. Write the closing remark in 1 paragraph that includes your future steps and current gaps to achieve your dream project.
\begin{thebibliography}{9}
\bibitem{texbook}
Some Author/Website Name (1986), Url to the Website. Some more related bib text
\end{thebibliography}
\end{document}

