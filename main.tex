\typeout{IJCAI-19 Multiple authors example}
\documentclass{article}
\pdfpagewidth=8.5in
\pdfpageheight=11in
% The file ijcai19.sty is NOT the same than previous years'
\usepackage{ijcai19}
% Use the postscript times font!
\usepackage{times}
\usepackage{soul}
\usepackage{url}
\usepackage[hidelinks]{hyperref}
\usepackage[utf8]{inputenc}
\usepackage[small]{caption}
\usepackage{graphicx}
\usepackage{amsmath}
\usepackage{booktabs}
\urlstyle{same}
% the following package is optional:
%\usepackage{latexsym} 

% Following comment is from ijcai97-submit.tex:
% The preparation of these files was supported by Schlumberger Palo Alto
% Research, AT\&T Bell Laboratories, and Morgan Kaufmann Publishers.
% Shirley Jowell, of Morgan Kaufmann Publishers, and Peter F.
% Patel-Schneider, of AT\&T Bell Laboratories collaborated on their
% preparation.

\title{My dream is to build - A city that actively controls its own wind environment}

\author{
Sourish Chatterjee$^1$\footnote{chatterjee.sourish@takenaka.co.jp}}
\begin{document}

\maketitle

\begin{abstract}
As our cities grow taller, we're facing new problems like dangerous wind gusts and stale, trapped air that make city life less pleasant and even unsafe. My dream project is to solve this by creating an "Active Wind Environment Control System." The core idea is that all the buildings in a city could work together to actively manage the wind. By using a network of sensors and an AI brain, the system would control small panels and fans on buildings to redirect strong winds, create gentle breezes where there are none, and help clear out air pollution. This isn't just a passive, one-time fix. It's a dynamic system where the city itself adapts to the weather in real-time to make it a safer, more energy-efficient, and healthier place to live.
\end{abstract}

\section{What is your Dream Project?}
My dream project is to create an "Active Wind Environment Control System." Put simply, I imagine a future where a whole city, with all its buildings, can work together as one big system to control how the wind flows through it in real-time. The plan would be to install small, smart devices—like movable panels or quiet fans—on the sides of buildings. These devices would be connected to a central AI that gets live data from wind sensors all over the city. The AI would then figure out the best way to adjust all the panels and fans to, for example, weaken a dangerous gust at a busy intersection or guide a refreshing breeze into a hot, stuffy plaza.
\par
Right now, most solutions to this problem are "passive," meaning they are designed once and then left alone, like the shape of the building itself or permanent wind-blocking walls. They can't adapt to changing weather. My idea is different because it's "active." The city would respond dynamically to the day's weather, almost like a living thing that can adjust itself for better comfort and safety.

\section{Why does your Dream Project matters?}
This project matters because it directly tackles two of the most critical issues in modern cities: public safety and energy consumption.First, it addresses safety. The dangerous wind gusts created between skyscrapers are a real problem that this system can solve.\cite{texbook} By actively controlling the airflow, it would create a fundamentally safer environment for everyone, from children to the elderly, preventing accidents before they can happen. Second, it contributes to energy conservation. Our cities consume massive amounts of power, and a significant portion of that goes to air conditioning. By using this system to intelligently manage natural airflow for cooling and ventilation, we could drastically cut down on that energy use. For me, this project is the perfect way to apply what I'm learning in engineering to a real-world problem, making a tangible difference in both the immediate safety and the long-term sustainability of our cities.

\section{What action you want to take to fulfill your Dream Project?}
As a mechanical engineering student, I see a clear path to begin working toward this dream, based on what I'm learning in my studies. My first step would be to dive deeper into CFD to create a simulation model that can accurately and quickly predict airflow around complex building layouts. This would be the core "brain" of the system. Next, I'd need to focus on the control algorithms, likely using AI and machine learning, to teach the system how to make the best decisions on its own. This is where fluid dynamics meets control engineering. Of course, as a mechanical engineer, I'm especially interested in designing the physical hardware. I want to develop durable, efficient, and quiet panels and fans that can be easily installed on buildings and withstand the elements for years. The final step would be to bring it all together by building a small-scale model of a city block, equipping it with my prototype devices, and testing the entire system in a wind tunnel to prove that it works and to refine it further.
\section{Conclusion}
This "Active Wind Environment Control System" is more than just a new piece of technology for buildings; it's a vision for a smarter, more responsive city. If this dream could be realized, our cities would no longer be just static collections of concrete and steel. Instead, they could become dynamic environments that actively look after the comfort and safety of their inhabitants. I know the challenges, like the computing power needed and getting different building owners to cooperate, are huge. Still, I believe the technology is moving in the right direction. My goal is to use my time at university to build the foundational skills in fluid dynamics, AI, and control systems needed to start turning this dream into a reality.
\begin{thebibliography}{9}
\bibitem{texbook}
Blocken, B., & Carmeliet, J. (2004). Pedestrian-level wind environment around buildings: Literature review and practical examples
\end{thebibliography}
\end{document}

